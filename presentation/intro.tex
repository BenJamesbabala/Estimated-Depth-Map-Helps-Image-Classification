\begin{abstract} 
We consider the problem of depth estimation from a single monocular image in this work.
It is a challenging task as no reliable depth cues are available, \eg, stereo correspondences, motions \etc.
Previous efforts have been focusing on exploiting geometric priors or additional sources of information, with all using hand-crafted features.
%
Recently, there is mounting evidence that features from deep convolutional neural networks (\cnn)  are setting new records for various vision applications.
On the other hand, considering the continuous characteristic of the depth values,  depth  estimations can be naturally formulated into a continuous conditional random field (\crf) learning problem.
Therefore, we in this paper present a deep convolutional neural field model for estimating depths from a single image, aiming to jointly explore the capacity of deep \cnn and continuous \crf.
Specifically, we propose a deep structured learning scheme which learns the unary and pairwise potentials of continuous \crf in a unified deep \cnn framework.

The proposed method can be used for depth estimations of general scenes with no geometric priors nor any extra information injected.
In our case, the integral of the partition function 
%
can be analytically calculated, thus we can exactly solve the log-likelihood optimization.  
Moreover, solving the MAP problem for predicting depths of a new image is highly efficient as closed-form solutions exist.
We experimentally demonstrate that the proposed method outperforms state-of-the-art depth estimation methods on both indoor and outdoor scene datasets.
\end{abstract}




\section{Introduction}
Estimating depths from a single monocular image depicting general scenes is a fundamental problem in computer vision, which has found wide applications in scene understanding, 3D modelling, robotics, \etc.  
It is a notoriously ill-posed problem, as one captured image may correspond to numerous real world scenes \cite{dcnn_nips14}.
Whereas for humans, inferring the underlying 3D structure from a single image is of little difficulties, it remains a challenging task for computer vision algorithms as no reliable cues can be exploited, such as temporal information, stereo correspondences, \etc. 
Previous works mainly focus on enforcing geometric assumptions, \eg, box models, to infer the spatial layout of a room \cite{Hedau_eccv10,Lee_nips10} or outdoor scenes \cite{Gupta_eccv10}.  
These models come with innate restrictions, which are limitations  to model only particular scene structures and therefore not applicable for general scene depth estimations. 
Later on, non-parametric methods \cite{depthTransfer_pami14} are explored, which consists of candidate images retrieval, scene alignment  and then depth infer using optimizations with smoothness constraints.
%
This is based on the assumption that scenes with semantic similar appearances should have similar depth distributions when densely aligned.
However, this method is prone to propagate errors through the different decoupled stages and relies heavily on building a reasonable sized image database to perform the candidates retrieval.
In recent years, efforts have been made towards incorporating additional sources of information, \eg, user annotations \cite{Russell_cvpr09}, semantic labellings \cite{Liu_cvpr12, Ladicky_cvpr14}. 
In the recent work of \cite{Ladicky_cvpr14}, Ladicky \etal have shown that jointly performing depth estimation and semantic labelling can benefit each other.
However, they do need to hand-annotate the semantic labels of the images beforehand as such ground-truth information are generally not available.
Nevertheless, all these methods use hand-crafted features.   

 

Different from the previous efforts, we propose to  
formulate the depth estimation as a deep continuous \crf learning problem, without relying on any geometric priors nor any extra information. 
Conditional Random Fields (\crf) \cite{Lafferty_icml01} are popular graphical models used for structured prediction. While extensively studied in classification (discrete) domains, \crf has been less explored for regression (continuous) problems.  
One of the pioneer work on continuous \crf can be attributed to \cite{ccrf_nips08}, in which it was proposed for global ranking in document retrieval.
%
Under certain constraints, they can directly solve the maximum likelihood optimization as the partition function  can be analytically calculated.
%
%
%
Since then, continuous \crf has been applied for solving various structured regression problems, \eg, remote sensing \cite{ccrf_ecai10,ccrf_aaai13}, image denoising \cite{ccrf_aaai13}.
Motivated by all these successes, we here propose to use it for depth estimation,
given the continuous nature of the depth values, and learn the potential functions in a deep convolutional neural network (\cnn).



%
Recent years have witnessed the prosperity of the deep convolutional neural network (\cnn). 
CNN features have been setting new records for a wide variety of vision applications \cite{astound_cvprW14}.
%
%
%
%
Despite all the successes in classification problems, 
deep \cnn has been less explored for structured learning problems, \ie, joint training of a deep \cnn and a graphical model, which is a relatively new and not well addressed problem.
To our knowledge, no such model has been successfully used for depth estimations. 
We here bridge this gap by jointly exploring \cnn and continuous \crf.
%
%
%
%



%
%
%
%
%
%
%



To sum up, we highlight the main contributions of this work as follows:
\begin{itemize}
\vspace{-.12cm} \item
We propose a deep convolutional neural field model for depth estimations by exploring \cnn and continuous \crf.  
%
Given the continuous nature of the depth values, the partition function in the probability density function can be analytically calculated, therefore we can directly solve the log-likelihood optimization without any approximations.  The gradients can be exactly calculated in the back propagation training.
Moreover, solving the MAP problem for predicting the depth of a new image is highly efficient since closed form solutions exist.
%
\vspace{-.12cm} \item
We jointly learn the unary and pairwise potentials of the \crf in a unified deep \cnn framework, which is trained using back propagation.
%
%
\vspace{-.12cm} \item
We demonstrate that the proposed method outperforms state-of-the-art results of depth estimation on both indoor and outdoor scene datasets.
\end{itemize}